\documentclass{sigchi}

% Use this command to override the default ACM copyright statement
% (e.g. for preprints).  Consult the conference website for the
% camera-ready copyright statement.


%% EXAMPLE BEGIN -- HOW TO OVERRIDE THE DEFAULT COPYRIGHT STRIP -- (July 22, 2013 - Paul Baumann)
% \toappear{Permission to make digital or hard copies of all or part of this work for personal or classroom use is      granted without fee provided that copies are not made or distributed for profit or commercial advantage and that copies bear this notice and the full citation on the first page. Copyrights for components of this work owned by others than ACM must be honored. Abstracting with credit is permitted. To copy otherwise, or republish, to post on servers or to redistribute to lists, requires prior specific permission and/or a fee. Request permissions from permissions@acm.org. \\
% {\emph{CHI'14}}, April 26--May 1, 2014, Toronto, Canada. \\
% Copyright \copyright~2014 ACM ISBN/14/04...\$15.00. \\
% DOI string from ACM form confirmation}
%% EXAMPLE END -- HOW TO OVERRIDE THE DEFAULT COPYRIGHT STRIP -- (July 22, 2013 - Paul Baumann)


% Arabic page numbers for submission.  Remove this line to eliminate
% page numbers for the camera ready copy 

%\pagenumbering{arabic}

% Load basic packages
\usepackage{balance}  % to better equalize the last page
\usepackage{graphics} % for EPS, load graphicx instead 
%\usepackage[T1]{fontenc}
\usepackage{txfonts}
\usepackage{times}    % comment if you want LaTeX's default font
\usepackage[pdftex]{hyperref}
% \usepackage{url}      % llt: nicely formatted URLs
\usepackage{color}
\usepackage{textcomp}
\usepackage{booktabs}
\usepackage{ccicons}
\usepackage{todonotes}

% llt: Define a global style for URLs, rather that the default one
\makeatletter
\def\url@leostyle{%
  \@ifundefined{selectfont}{\def\UrlFont{\sf}}{\def\UrlFont{\small\bf\ttfamily}}}
\makeatother
\urlstyle{leo}

% To make various LaTeX processors do the right thing with page size.
\def\pprw{8.5in}
\def\pprh{11in}
\special{papersize=\pprw,\pprh}
\setlength{\paperwidth}{\pprw}
\setlength{\paperheight}{\pprh}
\setlength{\pdfpagewidth}{\pprw}
\setlength{\pdfpageheight}{\pprh}

% Make sure hyperref comes last of your loaded packages, to give it a
% fighting chance of not being over-written, since its job is to
% redefine many LaTeX commands.
\definecolor{linkColor}{RGB}{6,125,233}
\hypersetup{%
  pdftitle={SIGCHI Conference Proceedings Format},
  pdfauthor={LaTeX},
  pdfkeywords={SIGCHI, proceedings, archival format},
  bookmarksnumbered,
  pdfstartview={FitH},
  colorlinks,
  citecolor=black,
  filecolor=black,
  linkcolor=black,
  urlcolor=linkColor,
  breaklinks=true,
}

% create a shortcut to typeset table headings
% \newcommand\tabhead[1]{\small\textbf{#1}}

% End of preamble. Here it comes the document.
\begin{document}

\title{Recruiting and Retaining Newcomers Using Online Bots}

%\numberofauthors{3}
%\author{%
 % \alignauthor{1st Author Name\\
  %  \affaddr{Affiliation}\\
   % \affaddr{City, Country}\\
    %\email{e-mail address}}\\
  %\alignauthor{2nd Author Name\\
   % \affaddr{Affiliation}\\
    %\affaddr{City, Country}\\
    %\email{e-mail address}}\\
  %\alignauthor{3rd Author Name\\
   % \affaddr{Affiliation}\\
    %\affaddr{City, Country}\\
    %\email{e-mail address}}\\
%}

\maketitle

%\begin{abstract}
 % UPDATED---\today. This sample paper describes the
  %formatting requirements for SIGCHI conference proceedings, and
  %offers recommendations on writing for the worldwide SIGCHI
  %readership. Please review this document even if you have submitted
  %to SIGCHI conferences before, as some format details have changed
  %relative to previous years. Abstracts should be about 150 words and
  %are required.
%\end{abstract}

%\keywords{Authors' choice; of terms; separated; by semi\-colons;
 % commas, within terms only; this section is required.}

%\category{H.5.m.}{Information Interfaces and Presentation
 % (e.g. HCI)}{Miscellaneous} \category{See
  %\url{http://acm.org/about/class/1998/} for the full list of ACM
  %classifiers. This section is required.}{}{}

\section{Introduction}
A main challenge faced by large scale collaborative projects is recruiting newcomers and retaining them~\cite{suh2009singularity}. The retention of newcomers is important because they impact the project's lifespan~\cite{halfaker2013making}. Newcomers can also expand the coverage of a project, importing novel perspectives. However, maintaining newcomers is  difficult. Wikipedia's volunteer workforce e.g., has been steadily decreasing since 2007~\cite{CHINOSAUR:venue}. 

Several platforms and workflows have attempted to tackle this problem. However, designing them is also complex and do not always translate into efficient solutions. For instance, several platforms use marketing schemes to import a flux of user~\cite{ribeiro2014modeling}. But, this does not necessarily translate into having long-term participants, especially because other more attractive platforms can again sway them away. Other approaches have created sandboxes, where newcomers can make safe contributions, and slowly become a part of the community~\cite{morgan2013tea}. However, this usually requires experienced volunteers to invest time engaging newcomers, limiting their own contributions. 
Other approaches have studied engaging newcomers with simple lightweight socialization processes~\cite{ciampaglia2015moodbar}. These approaches showcase how newcomers can be retained, while not imposing a large burden on others. However, these socialization processes generally operate within the group of users already present on the site~\cite{halfaker2013making}. This can limit the type of people who decide to initially take part, possibly influencing the platform's newcomber retention.


To help retain volunteers for large scale collaborative projects, such as Wikipedia, we present RoboWiki, a platform that leverages online bots, to actively recruit  new volunteers and then retains them via lightweight socialization processes. We speculate that by involving external people who publicly show a great interest or passion related to the project, we will be able to recruit them longer term than average new editors. We focus in particular on retaining newcombers for Wikipedia. We consider we can use Twitter to identify people with potential interest in editing a given Wikipedia article.  
Our system works as follows: An editor first presents the Wikipedia article for which she wants new volunteers, as well as the list of keywords that define the article. Secondly, the platform identifies people who appear to have an interest in the topic of the article based on simple keyword matching with their latests Tweets. The platoform then sends out bots to invite these people to edit and improve the Wikipedia article. If the person accepts, the same bot within Wikipedia provides a lightweight socialization process to help retain the new volunteers. We base the socialization process on MoodBar~\cite{ciampaglia2015moodbar}: newcomers  are requested for lightweight feedback about their editing experience. Experienced volunteers can see the feedback and provide guidance if needed. 

To understand the benefits and limitations of platforms which use automated social agents to recruit and retain new volunteers, we designed and conducted experiments on Twitter.  We deployed our platform publicly on Twitter, where our bots invited people to help edit Wikipedia articles. The bots recruited two different groups of individuals: people whose latest tweets showcased a potential interest for the Wikipedia article; and a set of randomly selected Twitter users (control group).  XX  volunteers responded to RoboWiki calls to edit; and YYY volunteers actually started editing the articles. These volunteers made XXX contributions (XXXX tweets,  XX favorites and retweets, edited XXX articles with XXX words). We found that the people with a potential interest in the article had more discussions on Twitter with 
RoboWiki, edited more articles; and remained longer on Wikipedia than the people without the previous interest (average XX months in comparison to XX days by control group). The people who were recruited by bots edited also a larger number of articles than the people who arrived to Wikipedia on their own. However, they also suffered initially more frustration editing (as expressed by their moodBar responses).

Together our results showcase how volunteers can be retained longer term  by recruiting people  who publically express an interest in the topic, and using socialization methods to guide their contributions and retain them longer term.


\begin{figure}
\centering
  \includegraphics[width=0.9\columnwidth]{figures/sigchi-logo}
  \caption{Insert a caption below each figure. Do not alter the
    Caption style.}~\label{fig:figure1}
\end{figure}

\subsection{References and Citations}



% Balancing columns in a ref list is a bit of a pain because you
% either use a hack like flushend or balance, or manually insert
% a column break.  http://www.tex.ac.uk/cgi-bin/texfaq2html?label=balance
% multicols doesn't work because we're already in two-column mode,
% and flushend isn't awesome, so I choose balance.  See this
% for more info: http://cs.brown.edu/system/software/latex/doc/balance.pdf
%
% Note that in a perfect world balance wants to be in the first
% column of the last page.
%
% If balance doesn't work for you, you can remove that and
% hard-code a column break into the bbl file right before you
% submit:
%
% http://stackoverflow.com/questions/2149854/how-to-manually-equalize-columns-
% in-an-ieee-paper-if-using-bibtex
%
% Or, just remove \balance and give up on balancing the last page.
%
\balance{}
% REFERENCES FORMAT
% References must be the same font size as other body text.
\bibliographystyle{SIGCHI-Reference-Format}
\bibliography{sample}

\end{document}

%%% Local Variables:
%%% mode: latex
%%% TeX-master: t
%%% End:
